\documentclass[conference]{IEEEtran}

\usepackage{cite}
\usepackage[pdftex]{graphicx}
% declare the path(s) where your graphic files are
\graphicspath{{images/}}

% Packages that are sometimes useful
%\usepackage{amsmath,amssymb,amsfonts}
%\usepackage{algorithmic}
%\usepackage{float}
%\usepackage{url}
    
\begin{document}

\title{Reviving the Lost Art of Combinator Graph Reduction}

\author{\IEEEauthorblockN{Daan de Graaf}
    \IEEEauthorblockA{\textit{Master's student Computer Engineering} \\
        \textit{TU Delft / Eindhoven University of Technology}\\
        Eindhoven, Netherlands \\
        d.j.a.degraaf@student.tudelft.nl / d.j.a.d.graaf@student.tue.nl}
}

\maketitle

\begin{abstract}
    % Rules of thumb
    % - 1 sentence background
    % - 1 sentence motive/problem
    % - 1 sentence objective
    % - 1 sentence approach/method
    % - 2 sentences support
    % - 1 sentence conclusion
    % - 1 sentence implication
    % - 1 sentence limitation
    This is the abstract.
\end{abstract}

\begin{IEEEkeywords}
    combinatory logic, graph reduction, lazy functional language, lazy evalution
\end{IEEEkeywords}

\section{Introduction}
% Why did I research this. Give the motive, usually a knowledge gap or problem
Lazy evaluation of programs is a powerful idea.
It allows a program to operate on infinitely large data structures, and if part of the computation turns out to be unnecessary, it can be much more efficient than a strict evaluation.
In practice however, delaying computation often introduces overhead, making languages based on lazy evaluation slower than their strict counterparts.
To compete with a strict implementation, a lazy runtime must maximize lazyness while minizing overhead.
It is a difficult balance, and over the years many different \textit{abstract machines} have been proposed\cite{kieburtz_g-machine_1985,fairbairn_tim_1987,burn_spineless_1988,koopman_fresh_1989,jones_implementing_nodate}.
One of the early approaches to implementating lazy functional languages is combinator graph reduction\cite{turner_new_1979}.

\textbf{TODO:} what is combinator reduction?

Combinator reduction was the runtime for David Turner's Miranda language\cite{turner_miranda_1985}, the popular choice for pure functional langauges at the time, and one of the few to be commercially supported.
Eventually the interest for Miranda and combinator graph reduction with it waned, and today Haskell is the dominant language.
Part of the reason may be that Miranda was closed source for 35 years, its source code was released only in 2020\cite{noauthor_open_2021}, but combinator reduction also deserves part of the blame.
Expressions are reduced in many small steps, so the per-step overhead greatly impacts runtime performance.
Every reduction step modifies the stack and usually also the graph, resulting in high memory traffic.
Another important issue is that the translation to combinators may result in much larger program.
For a lambda calculus expression of size $N$, the traditional bracket abstraction algorithm produces an output of $O(n^2)$ in the worst case, were other approaches such as supercombinators are linear.

These disadvantages have led to combinator graph reduction to be all but abandoned.
Recently however, a new algorithm for translation of lambda calculus expressions into combinatory logic has been proposed that is linear in the size of the input\cite{kiselyov_lambda_2018}.
While a clear improvement over previous translation strategies, it remains to be seen whether it can make the combinator reduction approach competitive with contemporary reduction engines.
This leads us to our first research question:

\textbf{RQ1:} How does Kiselyov's semantic translation compare to the classical bracket abstraction algorithm in terms of runtime performance of compiled programs?

We can take a set of benchmark programs and compile them with both algorithms, but to compare real-world execution time we also need a fast runtime to execute the compiled programs.

Alas, there is no performant combinator graph reduction engine on which we can execute programs compiled with Kiselyov's.
The now open sourced Miranda is a good starting point, but it is showing its age.
The 2020 port compiles cleanly on a recent GCC, but enabling optimizations is known to break the garbage collector.
Miranda's architecture is based on cells with a tag, head and tail as described in\cite{turner_new_1979}, but later work has shown that a tagless representation with direct pointers to code leads to better performance\cite{koopman_fresh_1989}.
In absence of an up to date reference on building a practical combinator reduction engine, we formlate the second research question:

\textbf{RQ2:} How is combinator graph reduction implemented efficiently on conventional hardware?

We evaluate the state of the art in combinator graph reduction, and build a fast implementation for a language that is simple but powerful enough to express our benchmark programs.
Using this engine, we can then answer RQ1.

% What did I research?
% - Research problem / question
%   * Definitions
%   * Choices
%   * Presuppositions
% - Why specifically this research question?
%   * Theoretical or societal relevance

% How did I research this?
% - Methods
% - Sub-questions

% How did I organize the report?
% - Structure overview

%NOTES:
%While initially a popular approach to lazy evaluation, combinator graph reduction fell out of fashion with the advent of new abstract machines based on supercombinators (G-machine, TIM).
%
%In the 1980s David Turner developed the language Miranda\cite{turner_miranda_1985}.
%At the time it was the popular choice for lazy functional programming, and one of few to be commercially supported.
%
%Lazy evaluation of programs is a powerful idea.
%It allows a program to operate on infinitely large data structures, and if part of the computation turns out to be unnecessary, it can be much more efficient than a strict evaluation.
%In practice however, delaying computation often introduces overhead, making languages based on lazy evaluation slower than their strict counterparts.
%To compete with a strict implementation, a lazy runtime must maximize lazyness while minizing overhead.
%
%Some of the early lazy functional languages were built on combinatory logic\cite{turner_new_1979,turner_miranda_1985}.
%Where the well-known lambda calculus has a concept of named variables that an evaluator must track and substitute values for in function application,
%in combinatory logic there are no bound variables, only a small .
%
%- Most important lazy functional language today is Haskell
%- It is built on top of the spineless tagless G-machine model
%- STG combines ideas from G-machine and TIM, both in the class of supercombinators
%- *What are supercombinators?*
%- Haskell inherits from Miranda, which was built on a very different model, combinator graph reduction
%- *What is Combinator Graph Reduction?*
%- Advantages of Combinators:
%* Small and fixed set of combinators. Reduction machine is simple, can be done in hardware.
%* Lambda bodies instantiated lazily, lazier than supercombinators
%- Disadvantages:
%* Reduction steps are small. Lot of transient memory traffic, high strain on garbage collector. Caching is also very coarse-grained.
%* Translation to combinators results in a larger program. (attack this with Kiselyov)
%- At the turn of the century, combinator graph reduction was all but abandoned.
%- Recently, a new translation strategy has been proposed for translating to combinators (cue Kiselyov).
%- After 35 years under a commercial license, Miranda was released under the open BSD license.
%- In this work we build on Miranda's design, incorporating the proposed semantic combinator translation algorithm and changes to the execution engine proposed in later papers.
%
%Research Questions:
%1. How is combinator graph reduction implemented efficiently on conventional hardware?
%2. How does Kiselyov's semantic translation compare to the classical bracket abstraction algorithm in terms of runtime performance of compiled programs?

\section*{Kiselyov's algorithms}
Describes the strict and lazy translation algorithms in a practical manner.

\section*{Engine implementation}
\subsection*{Miranda}
\subsection*{TIGRE}

\section*{Conclusion}
% Repeat the main question / objective of your research

% Give the answer to the question

% Why is this the answer? Present the main arguments

% What are the implications of this answer?

% Suggestions for future research

% FINAL CHECK: Can the conclusion be read independently?

\bibliographystyle{IEEEtran}
\bibliography{zotero}

\end{document}
