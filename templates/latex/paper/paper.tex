\documentclass[conference]{IEEEtran}

\usepackage{cite}
\usepackage[pdftex]{graphicx}
% declare the path(s) where your graphic files are
\graphicspath{{images/}}

% Packages that are sometimes useful
%\usepackage{amsmath,amssymb,amsfonts}
%\usepackage{algorithmic}
%\usepackage{float}
%\usepackage{url}
    
\begin{document}

\title{*Paper title}

\author{\IEEEauthorblockN{Daan de Graaf}
    \IEEEauthorblockA{\textit{Master's student Computer Engineering} \\
        \textit{TU Delft / Eindhoven University of Technology}\\
        Eindhoven, Netherlands \\
        d.j.a.degraaf@student.tudelft.nl / d.j.a.d.graaf@student.tue.nl}
}

\maketitle

\begin{abstract}
    % Rules of thumb
    % - 1 sentence background
    % - 1 sentence motive/problem
    % - 1 sentence objective
    % - 1 sentence approach/method
    % - 2 sentences support
    % - 1 sentence conclusion
    % - 1 sentence implication
    % - 1 sentence limitation
    This is the abstract.
\end{abstract}

\begin{IEEEkeywords}
    latex, IEEE, paper, template
\end{IEEEkeywords}

\section{Introduction}
% Why did I research this. Give the motive, usually a knowledge gap or problem

% What did I research?
% - Research problem / question
%   * Definitions
%   * Choices
%   * Presuppositions
% - Why specifically this research question?
%   * Theoretical or societal relevance

% How did I research this?
% - Methods
% - Sub-questions

% How did I organize the report?
% - Structure overview

\section*{Background}

\section*{Body section 1}
Reference to paper~\cite{xu_copy-and-patch_2021}.
\section*{Body section 2}

\section*{Conclusion}
% Repeat the main question / objective of your research

% Give the answer to the question

% Why is this the answer? Present the main arguments

% What are the implications of this answer?

% Suggestions for future research

% FINAL CHECK: Can the conclusion be read independently?

\bibliographystyle{IEEEtran}
\bibliography{zotero}

\end{document}
